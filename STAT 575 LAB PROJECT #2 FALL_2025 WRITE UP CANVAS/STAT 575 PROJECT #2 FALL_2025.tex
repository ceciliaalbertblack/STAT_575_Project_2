%% This document created by Scientific Notebook (R) Version 3.0
%\usepackage{sw20jart}
%\input{tcilatex}
%\input{tcilatex}
%\input{tcilatex}


\documentclass[12pt,thmsa]{article}
%%%%%%%%%%%%%%%%%%%%%%%%%%%%%%%%%%%%%%%%%%%%%%%%%%%%%%%%%%%%%%%%%%%%%%%%%%%%%%%%%%%%%%%%%%%%%%%%%%%%%%%%%%%%%%%%%%%%%%%%%%%%%%%%%%%%%%%%%%%%%%%%%%%%%%%%%%%%%%%%%%%%%%%%%%%%%%%%%%%%%%%%%%%%%%%%%%%%%%%%%%%%%%%%%%%%%%%%%%%%%%%%%%%%%%%%%%%%%%%%%%%%%%%%%%%%
\usepackage{amsmath}
\usepackage{eurosym}

\setcounter{MaxMatrixCols}{10}


\begin{document}


\begin{tabular}{l}
Prof. H. Bozdogan \ \ \  \\ 
Stat-575: Time Series Analysis%
\end{tabular}
\ \ \ \ \ \ \ \ \ \ \ \ \ \ \ \ \ \ \ \ \ \ 
\begin{tabular}{l}
FALL Semester: Oct. 23, 2025 \\ 
Due on or before Friday Nov. 4, 2025%
\end{tabular}

\smallskip

\section{ \ \ \ PROJECT \#2 \ TIME SERIES REGRESSION MODEL}

\smallskip

\subsubsection{INSTRUCTIONS:}

\smallskip

\begin{itemize}
\item Please SHOW ALL YOUR WORK. Please submit your write up, computational
modules and all your graphs with the

codes in A ZIPPED FILE such as JOHN\_DOE\_STAT575\_Proj\#2.zip to CANVAS.
\end{itemize}

\smallskip

\begin{itemize}
\item You can use any computational software such as Matlab, R, JMP, Python,
and SAS to be able to answer the following questions.
\end{itemize}

\vspace{1pt}

\subsubsection{OBJECTIVE}

\vspace{1pt}

The purpose of this project is to analyze a self-selected \textbf{time
series regression dataset} and to perform \textbf{subset selection of
predictor variables} using advanced covariance estimation methods. Students
will estimate models using heteroskedasticity- and
autocorrelation-consistent (HAC) covariance estimators and compare model
performance using three information-theoretic criteria: 
\begin{equation*}
\text{AIC},\quad \text{SBC (BIC)},\quad \text{and CICOMP (Consistent
Information Complexity)}.
\end{equation*}

Many time series data sets typically contains autocorrelation and/or
heteroskedasticity of unknown form and for statistical inference to model
such data sets it is essential to use \textsl{covariance matrix estimators}
that can consistently estimate the covariance of the model parameters. For
this reason, in the literature, there are several suitable \textsl{%
heteroskedasticity consistent} (\textsl{HC}) and \textsl{heteroskedasticity
and autocorrelation consistent} (\textsl{HAC}) \textsl{estimators} have been
proposed over the last 20 years.

\vspace{1pt}

\subsubsection{Dataset Requirements}

\vspace{1pt}

Each student must provide or construct their own dataset satisfying:

\vspace{1pt}

\begin{itemize}
\item The dataset must be in either \texttt{.csv} or \texttt{.xlsx} format.

\item It should include a numeric response variable $y_{t}$ (time series)
and a model matrix $\mathbf{X}_{t}$ of predictor variables.

\item The number of predictors must not exceed $k=10$.

\item The number of observations $n$ should be sufficiently large ($n>k$) to
estimate all candidate models.

\item Variables should be labeled clearly and described in a summary table
(see Table~below).
\end{itemize}

\vspace{1pt}\pagebreak

\section*{Modeling Framework}

\vspace{1pt}

You will fit the time series regression model 
\begin{equation*}
y_{t}=\beta _{0}+\sum_{j=1}^{p}\beta _{j}x_{tj}+\varepsilon _{t},\qquad
\varepsilon _{t}\sim \mathcal{N}(0,\sigma ^{2}),
\end{equation*}%
and estimate the covariance of the OLS coefficient vector $\hat{\beta}$
using a HAC estimator: 
\begin{equation*}
\text{Cov}_{\text{HAC}}=\hat{\Sigma}_{\text{HAC}}(w),
\end{equation*}%
where $w(\cdot )$ is the kernel weight function: 
\begin{equation*}
\text{Kernel}\in \{\text{TR},\text{BT},\text{PZ},\text{TH},\text{QS}\}.
\end{equation*}%
Here TR = Tukey, BT = Bartlett, PZ = Parzen, TH = Tukey\^{a}\euro
\textquotedblleft Hanning, and QS = Quadratic Spectral.

\vspace{1pt}

\textbf{Implementation Requirement:} Each student must use the provided
MATLAB module:

\begin{quote}
\texttt{dr\_TS\_AllSubsets\_HAC\_TOP\_EXPORT.m}
\end{quote}

to analyze their own dataset. This script will:

\begin{itemize}
\item Enumerate all possible subsets of predictors ($2^{p}$ models, $p\leq
10 $);

\item Fit each subset model via OLS and compute HAC covariance matrices;

\item Calculate AIC, SBC, and CICOMP for each subset;

\item Export a \texttt{.csv}, \texttt{.tex}, and \texttt{.rtf} summary of
the best model(s);

\item Generate comparative plots illustrating model complexity vs.
information criterion.

\item Which Kernel Covariance is the best choice for your dataset? Hint: You
can compare the score of the criteria on all the variables, saturated model.
\end{itemize}

\vspace{1pt}

\section*{Information Criteria}

\vspace{1pt}

For each subset model, compute the following: 
\begin{align*}
AIC& =n\log (2\pi )+n\log (\hat{\sigma}^{2})+n+2k, \\
SBC& =n\log (2\pi )+n\log (\hat{\sigma}^{2})+n+k\log n, \\
CICOMP& =n\log (2\pi )+n\log (\hat{\sigma}^{2})+n+k+2\log (n)C_{1F},
\end{align*}%
where $C_{1F}$ is the covariance complexity term derived from the
eigenvalues $\lambda _{1},\dots ,\lambda _{k}$ of the HAC covariance matrix
of $\hat{\beta}$: 
\begin{equation*}
C_{1F}=\frac{1}{4\bar{\lambda}^{2}}\sum_{i=1}^{k}(\lambda _{i}-\bar{\lambda}%
)^{2},\qquad \bar{\lambda}=\frac{1}{k}\sum_{i=1}^{k}\lambda _{i}.
\end{equation*}

\pagebreak

\section{Deliverables}

\vspace{1pt}

Each student must submit a concise technical report containing:

\begin{enumerate}
\item \textbf{Introduction and Data Description}

\begin{itemize}
\item Dataset source and description;

\item Table of variables (see example below);

\item Summary statistics and exploratory plots.
\end{itemize}

\item \textbf{Modeling and Results}

\begin{itemize}
\item Output from the module \texttt{dr\_TS\_AllSubsets\_HAC\_TOP\_EXPORT.m};

\item Tables of the best subsets under AIC, SBC, and CICOMP;

\item Visual comparison of each criterion across subset sizes.
\end{itemize}

\item \textbf{Discussion and Conclusions}

\begin{itemize}
\item Comparison among AIC, SBC, and CICOMP results;

\item Discussion of model parsimony and covariance complexity;

\item Insights into the predictive and structural properties of the selected
models.
\end{itemize}
\end{enumerate}

\section*{Example Variable Table}

\vspace{1pt}

\begin{tabular}{lll}
\textbf{Variable} & \textbf{Description} & \textbf{Type / Unit} \\ 
$y_{t}$ & Monthly retail sales & Thousands USD \\ 
$x_{1t}$ & Consumer confidence index & Index (0\^{a}\euro \textquotedblleft
100) \\ 
$x_{2t}$ & Average temperature & \^{A}%
%TCIMACRO{\U{b0}}%
%BeginExpansion
${{}^\circ}$%
%EndExpansion
C \\ 
$x_{3t}$ & Unemployment rate & \% \\ 
$x_{4t}$ & Average gasoline price & USD per gallon%
\end{tabular}

\vspace{1pt}

\section*{Submission Format}

\vspace{1pt}

\begin{itemize}
\item Submit one compressed folder (\texttt{.zip}) containing:

\begin{itemize}
\item Your dataset (\texttt{.csv} or \texttt{.xlsx});

\item MATLAB code and output files from \texttt{dr\_TS\_AllSubsets\_HAC\_TOP%
\_EXPORT.m};

\item Exported \texttt{.csv}, \texttt{.tex}, and \texttt{.rtf} tables;

\item Final report in \texttt{.pdf} format (compiled from LaTeX or using MS
Word).
\end{itemize}
\end{itemize}

\vspace{1pt}

\begin{itemize}
\item Include your name, course title, and date on the report title page.
\end{itemize}

\vspace{1pt}

\section*{Grading Rubric (100 points total)}

\vspace{1pt}

\begin{tabular}{lcc}
\textbf{Component} & \textbf{Description} & \textbf{Points} \\ 
Data Description & Dataset quality, variable documentation & 25 \\ 
Modeling and Computation & Proper use of HAC module and criteria & 40 \\ 
Analysis and Interpretation & Discussion, model comparison, reasoning & 25
\\ 
Report Format & Clarity, tables, plots, and LaTeX quality & 10 \\ 
\textbf{Total} &  & \textbf{100}%
\end{tabular}

\vspace{1pt}

\begin{center}
\textit{This project applies Bozdogan's Information Complexity (ICOMP)
framework to time series regression, illustrating the role of covariance
regularization and model parsimony in modern model selection.}
\end{center}

\vspace{1pt}

\begin{center}
\begin{tabular}{|l|}
\hline
If you have any questions, please ask me and Haoqi our GTA. \\ \hline
If you need more time, please let me know. \\ \hline
\end{tabular}
\end{center}

\end{document}
